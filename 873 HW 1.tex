\documentclass[11pt]{article}
\usepackage{geometry}                 
\geometry{letterpaper}                 
 \usepackage{graphicx}
\usepackage{amssymb,amsmath,amsfonts}

\newcommand{\rr}{{\mathbb R}}
 \newcommand{\F}{{\mathcal F}}
 \newcommand{\G}{{\mathcal G}}
 \newcommand{\pr}{{\mathbb P}}
\newcommand{\ex}{{\mathbb E}}
\newcommand{\ind}{{\mathbb I}}
\newcommand{\Q}{{\mathbb Q}}
\newcommand{\B}{{\mathcal B}}
\newcommand{\Fil}{{\mathcal F}}
\newcommand{\var}{{\text{Var}}}
\newcommand{\cov}{\text{Cov}}
\newcommand{\rto}{\Rightarrow}
\newcommand{\e}{{\varepsilon}}
\newcommand{\set}[1]{\left\{#1\right\}}
\newcommand{\abs}[1]{\left\vert #1\right\vert}
\newcommand{\norm}[1]{\left\| #1\right\|}
\newcommand{\charfun}{\phi_{n_k}(\frac{t}{s_n})}
\newcommand{\charind}{\frac{itX_{nk}}{s_n}}
\newcommand{\poiind}{\frac{it}{\sqrt{\lambda}}}
\newcommand{\posint}{[0,\infty]}
\parskip=10pt
  
  \begin{document}
 
\begin{center}
{\bf \large MATH 873 Homework 1}
\\
Sami Cheong
\\
\today
\end{center}
\begin{enumerate}
\item Let $Y$ be a modification of $X$, and suppose both processes have a.s. right-continuous sample paths. Then $X$ and $Y$ are indistinguishable.
\item[\textit{Pf.}] We want to show that 
\[
\pr\{X_t = Y_t, \forall t \in \posint\} = 1
\]
First, let us consider the sample paths at non-negative, rational time-points, i.e. Let
\begin{align*}
A & = \{\omega \in \Omega: X_t(\omega) = Y_t (\omega), \forall t \in \posint \cap \Q\}\\
\intertext{Since $\Q$ is a countable set, we can write}
A & ={\cap}_{t \in \posint \cap \Q} \{\omega \in \Omega: X_t(\omega) = Y_t (\omega) \}\\
\intertext{and also}
A^c & = \cup_{t \in \posint \cap \Q} \{\omega \in \Omega: X_t(\omega) \not= Y_t (\omega) \}
\intertext{Now, X is a modification of Y, so we have for a fixed $t$  and a.e. $\omega$, $ \pr(\{X_t \not= Y_t\}) = 0$. Moreover, since countable union of null sets is still a null set, thus}
0\leq \pr(A^c) & \leq \sum_{t \in \posint \cap \Q}\pr (\{X_t \not= Y_t\})  = 0, \text{(countable additivity)}\\
\intertext{which implies that $\pr(A) = 1$. This shows that $X$ and $Y$ agree on all the sample paths where $t \in \posint \cap \Q$. Also, we know that:}
\intertext{ 1) $X$ and $Y$ are right-continuous for a.e. $\omega \in \Omega$, and }
\intertext{ 2) $\Q$ is dense in $\rr$.}
\intertext{Thus, if $t \in \posint \cap \Q^c,$ by 1) and 2) we can find a sequence $\{t_n\} \subset  \posint \cap \Q$ such that $t_n \to t$ from the right, and}
X_t(\omega) &= \lim_{t_n \to t} X_{t_n}(\omega) =  \lim_{t_n \to t} Y_{t_n}(\omega) = Y_t(\omega)
\intertext{As a result, almost all the sample paths are now accounted for, which shows that $X$ and $Y$ are indistinguishable.}
\end{align*}

\item Let $\{\Fil_t\}$ be a filtration. Show that $\{\Fil_{t^+}\}$ is right-continuous.
\item[\textit{Pf.}] We want to show that $\Fil_{t^+} =\Fil_{t^{++}}.$ Notice that
\begin{align*}
\Fil_{t^{++}} & = \cap_{s > 0 }\Fil_{t^+ + s}\\
& = \cap_{s_1 > 0}(\cap_{s>0} \Fil_{t + s + s_1})\\
&= \cap_{s+s_1 > 0 } \Fil_{t + (s+s_1)}\\
&= \Fil_{t^+}.
\intertext{Thus, $ \F_{t^+}$ is right-continuous.}
\end{align*}
\item[3.] Let $\{\Fil_{n}, n =1,2 \dots\}$ be a decreasing sequence of sub-$\sigma$-algebras of $\Fil$ and let $\{X_{n}, \Fil_{n}\}$ be a backward submartingale. Then $l : = \lim_{n\to\infty} \ex [X_n] > -\infty$ implies that the sequence $\{X_n, n\geq 1\}$ is u.i.
\item[\textit{Pf.}] From the properties of backward submartingale,  we know that for $\forall n \geq 1, \ex [X_n] \leq \ex [X_{n-1}] \leq \dots \leq \ex [X_1]  < \infty$, and from the given assumption of $l$, we have $ \ex[X_n] \geq l.$ Also, notice that to this end
\begin{align*}
\ex{[|X_n|]} & = \ex{[X_n^+]} + \ex [X_n^-] \\
& = 2\ex{[X_n^+]} - \ex [X_n]\\
\intertext{\textbf{Claim:} $\{X_n^{+}\}$ is a backward submartingale}
\intertext{Proof of claim: Similar to the proof in Chung, Theorem 9.3.1, $\ex[X_n^{+}] < \infty$ is clear since $\ex[X_n^{+}] \leq \ex[|X_n|] < \infty$. Furthermore, since $(\cdot)^{+}$ is convex and increasing, Jensen's inequality implies that $X_{n+1}^{+} \leq (\ex(X_n|\Fil_{n+1}))^{+} \leq \ex(X_n^{+}|\Fil_{n+1}).$ Thus, we can now proceed to say that:}  
\ex{[|X_n|]} &\leq  2\ex [X_1^{+}] - \ex[ X_n] \\
&\leq \underbrace{2\ex |X_1| - l}_{\text{independent of }  n} < \infty\\
\intertext{This implies $\sup_{n\geq 1} \ex[|X_n|] < \infty$. Moreover, by Chebychev's inequality, we have}
\pr(\{|X_n| > K\}) &\leq \frac{\ex[|X_n|]}{K} \to 0 \text{ as } K \to \infty\\
\intertext{As a result, we have $\lim_{K\to\infty} \sup_{n\geq 1} \int_{\{|X_n| > K\}} |X_n| d\pr = 0,$ proving uniform integrability.}
\end{align*}

\end{enumerate}
\end{document}