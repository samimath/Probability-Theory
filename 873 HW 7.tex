\documentclass[11pt]{article}
\usepackage{geometry}                 
\geometry{letterpaper}                 
 \usepackage{graphicx}
\usepackage{amssymb,amsmath,amsfonts}

\newcommand{\rr}{{\mathbb R}}
 \newcommand{\F}{{\mathcal F}}
 \newcommand{\G}{{\mathcal G}}
 \newcommand{\LL}{{\mathcal L}}
 \newcommand{\M}{{\mathcal M}}
 \newcommand{\pr}{{\mathbb P}}
 \newcommand{\N}{{\mathbb N}}
\newcommand{\ex}{{\mathbb E}}
\newcommand{\ind}{{\mathbb I}}
\newcommand{\Q}{{\mathbb Q}}
\newcommand{\B}{{\mathcal B}}
\newcommand{\Fil}{{\mathcal F}}
\newcommand{\T}{{\mathcal T}}
\newcommand{\var}{{\text{Var}}}
\newcommand{\cov}{\text{Cov}}
\newcommand{\rto}{\Rightarrow}
\newcommand{\e}{{\varepsilon}}
\newcommand{\set}[1]{\left\{#1\right\}}
\newcommand{\abs}[1]{\left\vert #1\right\vert}
\newcommand{\norm}[1]{\left\| #1\right\|}
\newcommand{\charfun}{\phi_{n_k}(\frac{t}{s_n})}
\newcommand{\charind}{\frac{itX_{nk}}{s_n}}
\newcommand{\poiind}{\frac{it}{\sqrt{\lambda}}}
\newcommand{\posint}{[0,\infty]}
\parskip=10pt
  
  \begin{document}
 
\begin{center}
{\bf \large MATH 873 Homework 7}
\\
Sami Cheong
\\
\today
\end{center}
\begin{enumerate}
\item[(i)] Show that $f(x,y,z) = (x^2+y^2+z^2)^{-1/2}$ is harmonic.
\item[] By definition, $f$ is harmonic if $\Delta f = 0.$ 
Since 
\begin{align*}
\frac{\partial }{\partial x} f(x,y,z) & = -x (x^2+y^2+z^2)^{-3/2}\\
\frac{\partial^2 }{\partial x^2} f(x,y,z) & = -(x^2+y^2+z^2)^{-3/2} +3x^2(x^2+y^2+z^2)^{-5/2},
\intertext{by symmetry,}
\Delta f & = -3(x^2+y^2+z^2)^{-3/2} + 3(x^2+y^2+z^2)(x^2+y^2+z^2)^{-5/2} \\
& = 0.
\intertext{Thus, $f(W_t)$, where $W_t$ is a 3-d standard Brownian motion, is a local martingale.}
\end{align*}
\item[(2)] Since $W_t$ is normal with mean $\vec{0}$ and covariance $t I,$ and $M_t = f(W_t),$ we have
\begin{align*}
\ex[M_t^2] & = \int_{\rr^3} M_t^2 \frac{1}{(2\pi t)^{3/2}} \exp\{-\frac{\vec{W}_t \vec{W}'_t}{2t}\} d\vec{W}_t\\
&=\frac{1}{(2\pi t)^{3/2}} \int_{\rr^3} M_t^2 \exp\{-\frac{\vec{W}_t \vec{W}'_t}{2t}\} d\vec{W}_t \tag{1}
\intertext{Now, for convenience, let $t$ be fixed and denote $x= (W^{(1)}_t)$, $y= (W^{(2)}_t)$ and $ z=(W^{(3)}_t)$, and let $r = \sqrt{ x^2+y^2+z^2}$. Then $x = r \cos \theta \sin \phi$, and $y = r \sin \theta \sin \phi$, with $0 < r< \infty$, $ 0 < \theta < 2\pi$ and $0 < \phi < \pi.$ Notice that $M^2_t = \frac{1}{x^2+y^2+z^2}$, thus (1) can be expressed as: }
\ex[M^2_t] & = \frac{1}{(2\pi t)^{3/2} } \int_{0}^{\infty} \int_{0}^{\pi} \int_{0}^{2 \pi} \frac{1}{r^2} \exp\{\frac{-r^2}{2t}\} r^2 \sin \phi d\phi d\theta dr\\
&= \frac{1}{(2\pi t)^{3/2 } } \int_{0}^{\infty} \int_{0}^{\pi} \int_{0}^{2 \pi} e^{-r^2/2t} \sin \phi d\phi d\theta dr\\
& = \frac{1}{(2\pi t )^{3/2} }\underbrace{ \int_{0}^{\infty} e^{-r^2/2t} dr}_{=\sqrt{2\pi t}} \underbrace{ \int_{0}^{ \pi}\sin \phi d\phi}_{=1} \underbrace{ \int_{0}^{2\pi}    d\theta }_{= 2\pi}\\
&=\frac{\sqrt{t}}{(2\pi t)^{3/2} } (2\pi)^{3/2} =\frac{1}{t}
\end{align*}
\item[(iii)] Conclude that $M$ is not a martingale.
\item[Pf.] We can see that $\ex[M_t^2] \to 0$ as $t\to \infty$. This implies that $\ex [M_t] \to 0$ as well. But if $M_t$ is a martingale, then by definition there exists some constant $c$ s.t. $\ex[M_t] = c,$ so it cannot converge to 0. Moreover, under this assumption and by Jensen's inequality (since $x^2$ is convex for $x \geq 0$),
\[
\ex[M^2_t] \geq (\ex[M_t])^2 = c^2 > 0,
\]
thus we have a contradiction to the convergence of $\ex[M^2_t]$ as well, therefore $M_t$ is not a martingale. 
\end{enumerate}
\end{document}